\documentclass[../../../../main]{subfiles}
\graphicspath{{images/tandm/}}

\begin{document}

\paragraph{Tracking and Mapping}

Tracking and Mapping is somewhat a logical group that unites multiple algorithms under the hood. For instance, real-time Structure from Motion is also referred to as monocular \ac{SLAM} which is a group of tracking and mapping algorithms.

\ac{SLAM} approximately solves the problem of building and updating the map of environment; at the same time it computes and keeps track of the object position in the environment, e.g. robot \cite{springer_handbook_of_robotics}. 

Maps are used to determine a position in space and for a graphic representation of a terrain plan or for navigation. They are used to assess the actual location by recording information obtained from the perception form and comparing it with the current set of representations. The contribution of maps to the assessment of the current location in space increases with a decrease in the accuracy and quality of space sensing sensors. Maps reflect the type of space fixed at the time of their construction. It is not at all necessary that the kind of space will be the same at the time of using the maps.

The complexity of the technical process of determining the current location and the construction of the map is due to the low precision of the instruments participating in the calculation of the current location. The method of simultaneous localization and mapping is a concept that connects two independent processes into a continuous series of sequential computations. In this case, the results of one process are involved in the computation of another process.

\begin{figure} [ht!]
    \begin{center}
        \includegraphics[width=200pt]{slam}
        \caption{SLAM demonstration.}
        \label{fig:slam}
    \end{center}
\end{figure}

Monte Carlo methods and Extended Kalman Filter are often used in approximation process. These statistical approximation methods estimate object's pose and map environment parameters probability functions. Set-membership methods generate a set where object's pose and map approximation are contained. These methods are based on interval constraint propagation which is used for propagating uncertainties if the error data is represented as intervals. Already known Bundle adjustment can also be used for a better map reconstruction based on simultaneous estimations of object poses and landmarks positions.

\ac{SLAM} group of algorithms usually used in robotics, self-driven cars,and etc. Despite this fact, it is still an open field of research since current solutions are not ideal.

Methods for finding camera pose in an unknown environment were already developed by adapting monocular \ac{SLAM}. The two most known algorithms are EKF-SLAM and FastSLAM 2.0 \cite{fast_slam} but they are incremental which means that tracking and mapping operations are done one after another in a single pipeline. This leads to an update of every landmark and camera position at every frame. While this approach might be suitable for robotics it is not good enough for hand-held camera. This statement is argued by the fact that robot is able to move at a stable constant slow speed and for the hand it is not that easy to do. On the other hand, robot usually uses more data then just camera to build map with \ac{SLAM}. Errors generated during the work-flow of hand-held monocular \ac{SLAM} can cause the corruption of generated maps in the mentioned iterative methods.

\begin{figure} [ht!]
    \begin{center}
        \includegraphics[width=400pt]{SLAM_vs_PTAM}
        \caption{\ac{SLAM} and \ac{PTAM} work-flow comparison.}
        \label{fig:slam}
    \end{center}
\end{figure}

Since none of mentioned approaches was robust enough for hand-held case for the augmented reality usage, Georg Klein and David Murray developed \ac{PTAM}. It offers the approach where tracking and mapping processes are not linked together so that they can be executed in parallel. It makes tracking probabilistically independent from mapping thus any tracking method can be used, for instance, authors used coarse-to-fine with robust estimator.

\begin{figure} [ht!]
    \begin{center}
        \includegraphics[width=300pt]{ptam_initialization}
        \caption{\ac{PTAM} initialization.}
        \label{fig:slam}
    \end{center}
\end{figure}

\ac{PTAM} initialization should be performed before tracking begins. Initialization movement of the camera must include slight translation, just rotating camera will fail it. 
Decoupling of tracking and mapping allows algorithm not to process each frame since the algorithm is not iterative. Selective processing of frames is necessary for rejecting those with redundant information, for example, similar frames when camera is not moving. This selectivity helps to speed up the algorithm in comparison with iterative methods. The fact that key-frames are not following each other one by one means that processing can be done not within a strict real-time but should be finished by the time when new key-frame is added \cite{ptam}.

On the other hand, bundle adjustment replaces incremental mapping. It was chosen because of its successful usage in real-time visual odometry and tracking same as it is well-proven instrument in Structure from Motion. The initial map is built with five-point algorithm. Tracking is achieved by applying local bundle adjustment to the $n$ most recent camera poses where $n$ is set manually to achieve real-time performance. Building long-term map makes this algorithm different in comparison with other solutions. This map is built so that feature points are visited over and over again thus it can perform full-map optimization. In addition, 2D features tracking was not efficient enough for initialization step thus authors decided to rely on epipolar feature search.

On the contrary, there is another algorithm aiming to solve the same problem as \ac{PTAM}, its name is \ac{DTAM}. Authors claim that dense methods for tracking and mapping can provide more accurate and robust results in comparison with other world perception models based on features. In contrast with other real-time monocular \ac{SLAM} algorithms it builds dense 3D model and uses it for camera pose tracking \cite{dtam}.


\begin{figure} [ht!]
    \begin{center}
        \includegraphics[width=300pt]{dtam_dense}
        \caption{Estimating depth map from the bundle of images.}
        \label{fig:slam}
    \end{center}
\end{figure}

In fact, to track the camera motion at each frame, since it is iterative algorithm, it uses dense frame alignment to compare to the already build scene dense model. At each new frame the dense model is expanded and modified with the help of the dense textured depth maps. The texture-mapped model is generated from depth maps, they are created with the help of multi-view stereo and dense reconstruction from a set of images.

In addition, photometric information is gathered sequentially in order to solve depths maps incrementally. The authors have done this with the help of non-convex optimization and accelerated exhaustive search in order not to loose small details.

The way of how \ac{DTAM} tracks camera pose is more robust then its competitors and at least as good as feature-based ways of doing that. This happens because dense model is capable of handling occlusions and scale operations. Everything has its pros and cons so do the dense models, they perform worse because of motion blur or unfocused camera. Algorithm performs very good for local illumination changes while it is not robust for global illumination changes. Normalized cross correlation measure was also used for better handling of local and global illumination changes.

After bootstrap is done system becomes self-supported. Authors believe that people do not understand the power of dense methods for tracking and reconstruction even though they bring a lot of workarounds for feature-based algorithm's problems. It was also claimed that dense methods are the most robust and accurate because of the ability of matching at every pixel.


\end{document}