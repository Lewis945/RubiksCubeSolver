\documentclass[../../../../main]{subfiles}

\begin{document}

\paragraph{OpenCvSharp}

OpenCvSharp is one of many {\Csharp} OpenCV's wrappers. It has been created by the japanies developer nicknamed "shimat". This project is opensource and licensed under BSD 3-Clause License. This wrapper is good for its similarity to the classical C++ \ac{API} of \ac{OpenCV} though it easy to translate C++ samples to {\Csharp} without major changes. It was modelled close to the native C++ \ac{API} as much as possible.

Almost all classes of this library implements $IDisposable$ interface and that means there is no need to think of unmanaged resources as in C++, no memory leaks and etc. OpenCvSharp does not force users to stick to object-oriented work flow, if user wants to, it is possible to use functional style that is native for \ac{OpenCV}. Library also includes transformation of \ac{OpenCV} images to .NET bitmaps that simplifies image manipulations.

It can be used on both Windows and Linux but for Linux it is only possible via Mono Framework. In general, any platform that Mono supports can be used.

\begin{figure} [!ht]
  \centering    
    \lstset{style=sharpc}
        \begin{lstlisting}
        // Edge detection by Canny algorithm
        using OpenCvSharp;
        
        public class Program 
        {
            public static void Main() 
            {
                Mat source = new Mat("image.png", ImreadModes.GrayScale);
                Mat target = new Mat();
                
                Cv2.Canny(source, target, 50, 200);
                using (new Window("Source image", source)) 
                using (new Window("Target image", target)) 
                    Cv2.WaitKey();
            }
        }
        \end{lstlisting}
  \caption{OpenCvSharp Canny edge detection sample.}
  \label{opencvsharpcanny}
\end{figure}

\begin{figure} [!ht]
  \centering    
    \lstset{style=cplusplus}
        \begin{lstlisting}
        #include ...
        using ...
        
        int main(int arg, char** argv)
        {
            string image_name = "someimage.jpg";
        
            Mat source, source_gray;
            Mat target, detected_edges;
            
            source = imread(image_name.c_str(), IMREAD_COLOR);
            cvtColor(source, source_gray, COLOR_BGR2GRAY);
        
            Canny(source_gray, detected_edges, ...);
            target = Scalar::all(0);
        
            source.copyTo(target, detected_edges);
        
            namedWindow( "Display window", WINDOW_AUTOSIZE); 
            imshow("Source image", source);                
            imshow("Target image", target);
        
            waitKey(0);
            return 0;
        }
        \end{lstlisting}
  \caption{C++ OpenCV API Canny edge detection sample.}
  \label{cpluspluscanny}
\end{figure}

As it can be seen from figures \ref{opencvsharpcanny} and \ref{cpluspluscanny} \ac{API}'s are very similar but C++ requires more code to write for the same operation but it is quite easy to translate these samples in both ways due to their similarity.

\end{document}