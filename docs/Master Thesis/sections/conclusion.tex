\documentclass[../main.tex]{subfiles}

\graphicspath{{images/contribution/}}

\begin{document}

The current project is representing a deep research in the field of computer vision with its complex algorithms and stereo vision concepts. Kalman filter was analyzed for suitability to the current problem. The program was developed in {\Csharp}. It can extract data about a Rubik's cube, build cube's model, solve and render it. Even though the extraction part is not giving incredible results it works as expected.

\begin{figure} [ht!]
    \begin{center}
        \includegraphics[width=300pt]{program_window}
        \caption{Window of the program.}
        \label{fig:framework_program_window}
    \end{center}
\end{figure}

As can be seen from the figure \ref{fig:framework_program_window}, the program has two buttons for solving. They are "Next" and "Solve", next button provide step by step solution displaying though user can solve the cube sitting in front of computer. Solve button, on the contrary, tries to solve and display the result immediately, if the cube is correct and can be solved. There is a possibility to solve a cube using logged (line by line) solution formulas in the specific text area.

\end{document}