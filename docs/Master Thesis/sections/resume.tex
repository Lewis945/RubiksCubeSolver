\documentclass[../main]{subfiles}
\graphicspath{{images/}}

\begin{document}

Selles töös uuritakse kuidas arvuti nägemisega seotud algoritme on võimalik rakendada objektide tuvastuse probleemile. Täpsemalt, kas arvuti nägemist on võimalik kasutada päris maailma kombinatoorsete probleemide lahendamiseks. Idee kasutada arvuti rakendust probleemide lahendamiseks tulenes tähelepanekust et probleemide lahenduse protsessid on kõik enamasti algoritmid. Sellest võib järeldada, et arvutid sobivad algoritmiliste probleemide lahendamiseks paremini kui inimesed, kellel võib sama ülesande peale kuluda kordades kauem. Siiski ei vaatle arvutid probleeme samamoodi nagu inimesed ehk nad ei saa probleeme analüüsida. Niisiis selle töö panuseks saab olema erinevate arvuti nägemise algoritmide uurimine, mille eesmärgiks on päris maailma kombinatoorsete probleemide tõlgendamine abstraktseteks struktuurideks mida arvuti on võimeline mõistma ning lahendama.

Praegu on antud valdkonnas vähe materiali, mis annab hea võimaluse panustada sellesse valdkonda. Seda saavutatakse läbi empiirilise uurimise testide kogumiku kujul selleks, et veenduda millised lähenemised on kõige paremad. Nende eesmärkide saavutamiseks töötati läbi suur hulk arvuti nägemisega seotud materjale ning teooriat. Lisaks võeti ka arvesse reaalaja toimingute tähtsus, mida võib näha erinevate liikumisest struktuuri eraldavate algoritmide(SLAM, PTAM) õpingutest, mida hiljem edukalt kasutati navigatsiooni ja liitreaalsuse probleemide lahendamiseks. Siiski tuleb mainida, et neid algoritme ei kasutatud objektide omaduste tuvastamiseks.

See töö uurib, kuidas saab erinevaid lähenemisi kasutada selleks, et aidata vähekogenud kasutajaid kombinatoorsete päris maailma probleemide lahendamisel. Lisaks tekib selle töö tulemusena võimalus tuvastada objektide liikumist (translatsioon, pöörlemine), mida saab kasutada koos virutaalse probleemi mudeliga, et parandada kasutaja kogemust.  \vspace{5mm} \\ \textbf{Märksõnad}: Kombinatoorsed probleemid, Rubiku kuubik, Huvipunktide tuvastamine, PTAM, DTAM, SLAM, Jälgimine ning kaardistamine, Liikumisest struktuuri eraldamine, 3D rekonstruktsioon, Kalmani filter \\
\textbf{CERCS}: P170

\end{document}