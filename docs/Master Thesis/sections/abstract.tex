\documentclass[../main]{subfiles}
\graphicspath{{images/}}

\begin{document}

This thesis describes and investigates how computer vision algorithms and stereo vision algorithms may be applied to the problem of object detection. In particular, if computer vision can aid on puzzle solving. The idea to use computer application for puzzle solving came from the fact that all solution techniques are algorithms in the end. This fact leads to the conclusion that algorithms are well solved by machines, for instance, a machine requires milliseconds to compute the solution while a human can handle this in minutes or hours. Unfortunately, machines cannot see puzzles from human perspective thus cannot analyze them. Hence, the contribution of this thesis is to study different computer vision approaches from non-related solutions applied to the problem of translating the physical puzzle model into the abstract structure that can be understood and solved by a machine.

Currently, there is a little written on this subject, therefore, there is a great chance to contribute. This is achieved through empirical research represented as a set of experiments in order to ensure which approaches are suitable. To accomplish these goals huge amount of computer vision theory has been studied. In addition, the relevance of real-time operations was taken into account. This was manifested through the Different real-time Structure from Motion algorithms (SLAM, PTAM) studies that were successfully applied for navigation or augmented reality problems; however, none of them for object characteristics extraction.

This thesis examines how these different approaches can be applied to the given problem to help inexperienced users solve the combination puzzles. Moreover, it produces a side effect which is a possibility to track objects movement (rotation, translation) that can be used for manipulating a rendered game puzzle and increase interactivity and engagement of the user. \vspace{5mm} \\ \textbf{Keywords}: combination puzzles, Rubik's cube, feature detection, PTAM, DTAM, SLAM, Tracking and Mapping, Structure from Motion, 3D reconstruction, Kalman filter \\
\textbf{CERCS}: P170

\end{document}