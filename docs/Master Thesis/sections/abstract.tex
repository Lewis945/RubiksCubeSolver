\documentclass[../main]{subfiles}
\graphicspath{{images/}}

\begin{document}

This thesis describes and investigates how computer vision algorithms, moreover stereo vision algorithms, may be applied to the problem of objects detection extrapolated on game puzzles detection, extraction and solving. The study examines different approaches from non related approaches applied to the given problem, as currently there is a little written on this subject. This is achieved through empirical research represented as a set of experiments in order to ensure whether approaches are suitable. To accomplish these goals huge amount of computer vision theory has been analyzed. The topic was chosen as the relevance of game puzzles is always the question because some new similar but different puzzles appear all the time and the relevance of real-time processing. Different real-time Structure from Motion algorithms (SLAM, PTAM) were successfully applied for navigation or augmented reality problems but none of them for objects tracking. This thesis examines how these different approaches can be applied for the given problem to help uninformed users be on easy touch with logical games. Moreover, it produces a side effect which is possibility to track objects movement (rotation, translation) that can be used for manipulating a rendered game puzzle and increase interactivity and user engagement.

\end{document}