\documentclass[../main]{subfiles}

\begin{document}

Nowadays computer vision is a widely used technology. It has started developing since the late 1960s but a real leap into the future has been done in 1990s, when the algorithms of 1960s got efficient implementations and were widely deployed due to the technical progress of 1990s. On the other side, hardware world was rapidly evolving which resulted to the current situation in computer vision field of research. Today the level of development (\ac{IoT} devices, smart phones, tablets and etc.) allow to integrate computer vision tightly into humans life. There can be enormous amount of examples like parking systems when there is a camera mounted on a barrier at an entrance of a parking lot and a car number is recognized with this camera to decide whether the barrier will be opened or not; cameras mounted on the roads for catching traffic violations; OCR technologies for analyzing text on images which allows to take a picture of a document and receive a typed version of it without typing by hand; applications for finding pedestrians on a road; applications for face recognition, they help to find criminals in the streets, airports or any public place where cameras are mounted; this list can be almost infinite.

The demand for these technologies by society motivates researches to develop new techniques, improve algorithms that already exist and were in use for the past decade. The previous decade has changed the world drastically, there have been created dozens of new concepts like social media, video platforms, 3d printing, cameras in mobile phones and etc. All of these have made a great influence on a human life. People's appetite is growing significantly and therefore new entertainment systems appear. Microsoft created Kinect, Sony has Move, Nintendo has Remote Plus and such kind of tools are based on computer vision however the technologies beneath them are not ideal. they need to gather a lot of data with a good quality, proceed complex computations and still the result is not always as good as expected. Taking into account these facts, there was a motivation to create new algorithms and to make attempts for obtaining as much data as possible of existing hardware. 

\subsection{Problem statement}

Tools mentioned above allow to capture objects movements, recognize patterns and etc. With their help it is possible to analyze density of an image, create a 3d reconstruction, control game objects with our hands moving, shaking and many other inter-activities became real. By looking closer at Kinect and Move it is clearly seen that they are just an expensive set of sensors for computer vision. In addition to multiple cameras they have infrared sensors, color sensors, depth sensors, microphones and etc. All these sets of sensors allow gathering a huge amount of data. From this perspective, even the simplest algorithms may provide quite a good result. In contrast, if we do not have such an expensive hardware the algorithmic part takes the leading role.

There are dozens of different algorithms already invented, some of them are applicable to the nowadays situations and some of them need an improvement. Algorithms could have specific purpose at the beginning but with a time they might be applied to other fields as well as becoming a real breakthrough there. This thesis will focus on experiments with the algorithms that aim at achieving the same result as with Kinect or another expensive hardware while having a simple monocular camera. These experiments are needed for understanding whether it is possible to build a solution for successful tracking of objects and reflecting all the changes (rotations, translations) from the real object onto its virtual model (mapping) on a computer. All existing algorithms have different purposes and can't be easily applied to the mentioned problem out of the box.

Towards this end, main research goals of this thesis are:
\begin{enumerate}
\item  Analyze every existing algorithm that can be applied to this problem.
\item  Take the parts of the theory beneath those algorithms and combine them together.
\item  Design a new approach for a mentioned problem based on the main seeds of the already existing algorithms. 
\end{enumerate}

\subsection{Contributions}

Methodology of this thesis is so called empirical research and it targets to achieve next goals:

\begin{itemize}
  \item Examine feature detection methods in computer vision.
  \item Examine feature classification methods in computer vision.
  \item Examine theory of stereovision (camera calibration, epipolar geometry, triangulation, pose estimation) in computer vision.
  \item Analyze theory beneath "Structure from Motion" approach.
  \item Analyze theory beneath "Tracking and Mapping" approaches.
  \item Build a framework for tracking and recognizing objects using monocular camera. Since Rubik's cube was picked as a simple shape object, using captured data it will be reconstructed and solved.
\end{itemize}

\subsection{Road map}

The rest of the master thesis has the following chapters: 

\chapref*{chap:background}: Presents an overview of all the related theory in computer vision: basic theory as feature detection/classification, basics of stereo vision on example if a camera calibration, "Structure from Motion" concept overview, "Tracking and Mapping" concept and its derivatives overview, Kalman filter explanations.

\chapref*{chap:experiments}: Contains overview of all experiments with algorithms and their implementations from the \chapref*{chap:background}. Includes descriptions of test cases, used approaches, used ready-implementations or self written ones. Covers testing and validating results.

\chapref*{chap:contribution}: Covers a solution of the mentioned problem based on the results of the experiments. Self written PTAM based approach covered and reasoned. Results are validated, pros and cons are presented.

\chapref*{chap:conclusion}: Overviews the results of the thesis and perspectives of designed approach.

\end{document}