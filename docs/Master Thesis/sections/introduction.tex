\documentclass[../main]{subfiles}

\begin{document}

\TODO{What is it in simple terms (title)?}
\TODO{Why should anyone care?}
\TODO{What was my contribution?} 
\TODO{What you are doing in each section (a sentence or two per section)}

\subsection{Problem statement}

\subsection{Contributions}

\subsection{Road map}

Semester 1

This project based on the previous work of Artjom Lind, Rauno Ots, Vilen Looga, and Murad Kamalov at the University of Tartu, Estonia. Author’s goal was to create software that can take image, taken on the mobile phone camera, with Sudoku as an input, process it and solve Sudoku, which is a content of the image. Researches of the base work had no experience working with image recognition; author of the current project is in the same situation. Another goal of researchers was to apply knowledges gained from data mining course. They figured out how to use classification algorithms for numbers recognition, tried some of them, and picked the one that performs the best for their purposes. Another task for them was to find out how many data is needed to train classification algorithms. This project gave all the developers opportunity to gain new knowledges, clear understanding of things which are hidden under the mysterious OCR, and digging dipper into data mining course in order to understand how classification works. As a result, software for solving Sudoku using received content from an image was developed. Since the previous implementation was created using opencv 2.4 that was using c++ methods as they are just in python syntax but now opencv 3.0 is released, it is tightly coupled with NumPy library and a lot of useful features were encapsulated it was decided to write the new version of the program.

Semester 2

This project based on the previous work of Jay Hack and Kevin Shutzberg at Stanford University, Stanford, USA. Author’s goal was to create software that can transfer state of a Rubik’s cube (highly mathematical puzzle) to a computer from a computer vision perspective. They decided that this problem is good for convolutional neural networks due to cube’s faces are highly structured and have distinct pattern and textures.
They presented an approach that is based on two separately trained convolutional networks. One network is used for localization of the cube on a video stream and another is used to detect centers of the cube faces from cropped images. Furthermore, the presented some techniques from classical computer vision: featuring SLIC superpixel segmentation, projective transforms and color histogram classifications. In addition, they used technique for gathering large amount amounts of labeled data for similar problems that is based on iterative training and supported by labeling procedure. In the end, they analyze the performance gains. Authors have their code hosted on GitHub, it is fully open source. Link: https://github.com/jayhack/ConvCube.

Semester 3

This project is based on multiple projects: the first one is Sudoku Solver by Artjom Lind and his fellows and my one approach of the same task; the second one is my own approach of Rubik’s cube detection and extraction of data for cube reconstruction and solving; and the last one but not the least one is Bootstrapping planar AR and tracking without markers [1]. Since in the previous work it was managed to extract cube faces from the video stream this time the goal was set to do the same but with different approach that will help me to expand abilities of the program, for instance, make the program to be able to extract, track, and analyze any figure, not only a cube. To complete my goals I analyzed different source and PTAM (Parallel Tracking and Mapping for Small AR Workspaces) [2] was somehow an answer to the current problem. Since this algorithm was too complicated to write it by myself and running C++ Linux code under .NET platform was not an easy task too there was found another approach [1] which behaves the same as PTAM but in much simpler way. In this project running this algorithm was a goal and it was successfully gained. Since this project is a part of the master thesis, full project goals are covered in this paper.

\end{document}